\begin{abstract}
    En este trabajo implementaremos un Ray Tracer basado en las guías de Shirley
    \footnote[1]{https://raytracing.github.io/books/RayTracingInOneWeekend.html} capaz de renderizar
    distintos objetos (esferas,
    cubos, planos y mallas de triángulos) y materiales (dieléctricos, metales y colores difusos).
    Además veremos una de las estructuras más utilizada para optimizar el renderizado cuando tenemos
    muchos objetos en una escena, el KD Tree.
\end{abstract}


\section{Introducción} \label{sec:introduccion}

En el mundo de la computación gráfica, existen distintos algoritmos para renderizar objetos 3D
en imágenes 2D. Se pueden clasificar en 2 categorías: \textit{image-order
rendering} y \textit{object-order rendering}.
En \textit{object-order rendering}, se considera cada objeto de una escena y se modifican los
pixeles que este influye.
Por otro lado, en \textit{image-order rendering}, se recorre cada pixel de la imagen, y se
calcula su valor considerando a todos los objetos que influyen en el.

El Ray Tracing es un ejemplo de un algoritmo de \textit{image-order rendering}.
Para generar una imagen, se lanzan \textit{rayos} hacia cada pixel y se calcula el color del
mismo como una combinación de todos los objetos con los que este rayo colisiona.